% ========================================================================
% 
% DIPLOMA THESIS - Mask R-CNN in GRASS GIS
% 
% Ondřej Pešek
% 
% ========================================================================

\documentclass[
  12pt,         			% velikost základního písma je 12 bodů
  a4paper,      			% formát papíru je A4
  oneside,       			% Oboustranný tisk
  pdftex,				    % překlad bude proveden programem 'pdftex' do PDF
  english,
  %draft
]{report}       			% dokument třídy 'zpráva'


\newcommand{\Fbox}[1]{\fbox{\strut#1}}

\usepackage[english, czech]{babel}	% použití češtiny, angličtiny
\usepackage[utf8]{inputenc}			% kódování zdrojových souborů je UTF8

\usepackage[square,sort,comma,numbers]{natbib}

\usepackage{caption}
\usepackage{subcaption}
\usepackage{listings}

\usepackage[dvipsnames]{xcolor}
\definecolor{light-gray}{gray}{0.95}

\captionsetup{font=small}
\usepackage{enumitem} 
\setlist{leftmargin=*} % bez odsazení

\makeatletter
\setlength{\@fptop}{0pt}
\setlength{\@fpbot}{0pt plus 1fil}
\makeatletter

\usepackage[dvips]{graphicx}   
\usepackage{color}
\usepackage{transparent}
\usepackage{wrapfig}
\usepackage{float} 

\usepackage{cmap}           
\usepackage[T1]{fontenc}    

\usepackage{textcomp}
\usepackage[compact]{titlesec}
\usepackage{amsmath}
\addtolength{\jot}{1em} 

\let\counterwithout\relax
\let\counterwithin\relax
\usepackage{chngcntr}
\counterwithout{footnote}{chapter}

\usepackage{acronym}

\usepackage[
    unicode,                
    breaklinks=true,        
    hypertexnames=false,
    colorlinks=true, % true for print version
    citecolor=black,
    filecolor=black,
    linkcolor=black,
    urlcolor=black
]{hyperref}         

\usepackage{url}
\usepackage{fancyhdr}
% \usepackage{algorithmic}
\usepackage{algorithm}
\usepackage{algcompatible}
\renewcommand{\ALG@name}{Pseudocode}% update algorithm name
\def\ALG@name{Pseudocode}

\usepackage[
  cvutstyle,          
  diploma   
]{thesiscvut}


\newif\ifweb
\ifx\ifHtml\undefined % mimo HTML.
    \webfalse
\else % v HTML.
    \webtrue
\fi 

\renewcommand{\figurename}{Obrázek}
\def\figurename{Obrázek}

\lstdefinestyle{python}{
   language=python,
   basicstyle={\footnotesize\ttfamily},
   keywordstyle=\color{blue}\ttfamily,
   stringstyle=\color{green}\ttfamily,
   commentstyle=\color{brown}\ttfamily,
   showstringspaces=false,
   morekeywords={True, False, sqrt}
}

\renewcommand\lstlistingname{Pseudocode}
\renewcommand*{\lstlistlistingname}{Content of pseudocodes}

\usepackage{dirtree}

\lstset{
	extendedchars=true,
	literate={á}{{\'a}}1}

\makeatletter
\newcommand\footnoteref[1]{\protected@xdef\@thefnmark{\ref{#1}}\@footnotemark}
\makeatother

\usepackage{tikz}
\usetikzlibrary{arrows.meta, shapes}
\tikzset{%
  >={Latex[width=1mm,length=1mm]},
  % Specifications for style of nodes:
            base/.style = {rectangle, rounded corners, draw=black,
                           minimum width=4cm, minimum height=1cm,
                           text centered, font=\sffamily},
  activityStarts/.style = {base, fill=blue!30},
       startstop/.style = {base, fill=red!30},
    activityRuns/.style = {base, fill=green!30},
    test/.style = {base, diamond, aspect=2, text width=8em, fill=yellow!30},
         process/.style = {base, minimum width=2.5cm, fill=orange!15,
                           font=\ttfamily},
}

\usepackage[justification=centering]{caption}

% ========================================================================
% Definice informací o dokumentu
% ========================================================================

% název práce
\nazev{Mask R-CNN in GRASS GIS}
{Mask R-CNN v prostředí GRASS GIS}

% jméno a příjmení autora
\autor{Bc. Ondřej}{Pešek}

% jméno a příjmení vedoucího práce včetně titulů
\garant{Ing.~Martin~Landa,~Ph.D.}

% označení oboru studia
\oborstudia{Geomatics}{}

% označení ústavu
\ustav{Department of Geomatics}{}

% rok obhajoby
\rok{2018}

% měsíc obhajoby
\mesic{}

% místo obhajoby
\misto{Praha}

% abstrakt
\abstrakt 
{TODO: THIS WILL BE ABSTRACT.}
{A TOHLE TAKY.}


% klíčová slova
\klicovaslova
{GIS, GRASS GIS, Python, artificial neural networks, Mask R-CNN, instance segmentation}
{GIS, GRASS GIS, Python, umělé neuronové sítě, Mask R-CNN, instanční segmentace}

% ========================================================================
% Nastavení polí ve vlastnostech dokumentu PDF
% ========================================================================
\nastavenipdf

% začátek dokumentu
\begin{document}

\catcode`\-=12  % pro vypnutí aktivního znaku '-' používaného např. v \cline 

% aktivace záhlaví
\zahlavi

% předefinování vzhledu záhlaví
\renewcommand{\chaptermark}[1]{%
	\markboth{\MakeUppercase
	{%
	\thechapter.%
	\ #1}}{}}

% vysázení přebalu práce
%\vytvorobalku

% vysázení titulní stránky práce
\vytvortitulku

% Vysázení listu zadani
\stranka{}%
	{\includegraphics[scale=0.7]{./pictures/zadanidp.pdf}}%\sffamily\Huge\centering\ }%ZDE VLOŽIT LIST ZADÁNÍ}%
	%{\sffamily\centering Z~důvodu správného číslování stránek}

% vysázení stránky s abstraktem
\vytvorabstrakt

% vysázení prohlaseni o samostatnosti
\vytvorprohlaseni

% vysázení poděkování
\stranka{%nahore
       }{%uprostred
       }{%dole
       \sffamily
	\begin{flushleft}
		\large
		\MakeUppercase{Acknowledgement}
	\end{flushleft}
	\vspace{1em}
		%\noindent
	\par\hspace{2ex}
	{TODO: \\
	THIS IS DEDICATED TO\\
	SOMEONE ELSE, NOT YOU.}
}

% vysázení obsahu
% \setcounter{tocdepth}{1}
\obsah

% vysázení seznamu obrázků
\seznamobrazku

% vysázení seznamu tabulek
% \seznamtabulek

% vysázení seznamu ukázek kódu
\cleardoublepage
\thispagestyle{empty}
 \lstlistoflistings
\newpage

% jednotlivé kapitoly
\chapter{Introduction}
\label{intro}

In the last years, the field of computer science is permanently shaken by one 
term: Artificial neural networks (\zk{ANN}s). Some people perceive it almost
as a magical formula. And even though \zk{ANN}s are not a spell able to solve
everything, they have wide applications. Their applications are in finance,
data mining, language recognition, computer vision and many more.

Another term that can be heard more and more is \textit{the information age}.
The~availability of computers and the growth of memory limits result in huge
amounts of data.

The huge amount of data is a fuel for \zk{ANN}s. One hundred years ago, an
artificial intelligence was a phantasmagoria of science-fiction writers. Fifty
years ago, it was an idea facing only derision. Twenty-five years ago, a design
doomed to failure due to a lack of training data. Twelve years ago, a bold idea
of few men. Five years ago, an earthquake in each branch of the field of
computer science.

The data availability and their opening in the field of geomatics, as well as
the~computer performance, open doors for the usage of \zk{ANN}s in geographic
information systems (\zk{GIS}). Results of a special type of \zk{ANN}s called
convolutional neural networks (\zk{CNN}s) promises a lot in computer vision and
therefore also in \zk{GIS}, in tasks of detection and classification.

Chapter \ref{cnn} will briefly introduce the~theory behind \zk{CNN}s. In the
first part, the~history and the motivation behind them will be described. The
second part covers multiple types of layers used in \zk{CNN}s. Few pioneering
architectures will be covered.

Chapter \ref{image-ann} will introduce the term computer vision. Various tasks
of computer vision will be named and few breakthrough architectures not 
mentioned in the \zk{CNN} chapter will be described there. The research which 
concluded in the selection of Mask \zk{R-CNN} as the architecture for the 
implementation will be depicted in this chapter.

Chapter \ref{technologies} will describe some of the most important 
technologies used during
the~above-mentioned implementation.

The implementation will be the topic of chapter \ref{implementation}. It will 
summarize the motivation behind the architecture choose and code decisions, and
then describe the~uppermost parts of the code. The practical part of the~thesis
is exactly this implementation.

\chapter{Convolutional neural networks}
\label{cnn}

\setcounter{secnumdepth}{1}

Although it is assumed that the reader has sufficient prior knowledge of artificial neural networks and convolutional neural networks, this part briefly introduces convolutional neural networks, their layer types and few selected architectures. 

For better understanding of the topic, it is recommended to take a look at the holy book of deep learning, \cite{dl}.

\section{Introducing convolutional neural networks}
\label{understanding-cnn}

If you try to find an introduction to \zk{CNN}s on the internet, you may bump into a common statement that \zk{CNN}s are neuroscience-based deep neural networks using convolution and presumpting the input is an image. It is not exact. 

Though images are the most common input, according to \cite{dl}, \zk{CNN}s presumpt the input has a grid-like topology; apart from the computer vision, other applications include  for example natural language processing (as in \cite{cnn-nlp}) or anything representable as a grid-like topology (audio waveform as 1-D grid, RGB images as multichannel 2-D, CT scan as 3-D, etc.). 

A paradox inexactness is the term \textit{convolution} as in mathematical meaning, many \zk{CNN}s implement cross-correlation instead of real convolution. Cross-correlation may be seen as convolution without a kernel flipping. The reader can get more mathematical insight about the difference and harmlessness of this change from \cite{dl}. 

It is true that \zk{CNN}s are based on a neuroscience. They are inspired by Nobel prize laureates Hubel's and Wiesel's research on mammalian vision systems (firstly cats in \cite{hubel-cats1} and \cite{hubel-cats2}, later monkeys in \cite{hubel-monkeys}). Hubel and Wiesel found that some neurons (sorted in columns) strongly respond to specific to specific edge-like patterns but just a bit to other patterns. 

The eye stimulus on the retina is transferred through the optic nerve and the lateral geniculate nucleus into the primary visual cortex (sometimes referred to as V1), a part of the visual cortex located in the posterior pole of the occipital lobe. The primary visual cortex is organized in a 2-D spatial map representing visual stimuli from the retina and contains two cell types, simple cells and complex cells. Simple cells purpose is to compute a linear function (although some counterarguments against the linearity have been raised, see \cite{simple-cells}) of the image in a spatially localized field, while complex cells operations are to some extent position and lighting invariant. 

\section{Layer types}
\label{layers}

While the common approach in image processing of classical \zk{ANN}s is to vectorize the input, \zk{CNN}s emulate the neuroscientific approach outlined in chapter \ref{understanding-cnn} using feature maps (each color channel is a feature map). \zk{CNN}s profit from the fact that pixels in an image are ordered according to some structure. That allows neurons in layers to be connected just to certain region instead of heavily arduous fully-connected architecture. 

\zk{CNN}s consist from many layers with different functions. In the following, individual layer types are briefly described. 

\subsection{Convolutional layers}
\label{conv-layers}

The first layers through which is the input passed are convolutional layers. So, firstly, what is the convolution? 

In the geomatics field, we very often encounter the term \textit{kernel}. Kernel can be seen as a matrix (or a window) sliding across all the image pixels. The pixels contained in this window are a receptive field. As both the kernel and the receptive field are matrices of the same shape, in each position element wise multiplication is computed and outputed as an output matrix element. Because after such a filtering, the output matrix contains a 2-D activation map (a map where each position values say with witch probability the requested feature is on that position in the original image), the output is called a feature map. Kernels / filters are the subject of training. 

In case of stride 1 and without zero-padding, the feature map is naturally of shape $[original\_width - kernel\_width + 1] \times [original\_height - kernel\_height + 1]$. An example may be seen in figure \ref{fig:conv}. 

\begin{figure}[H]
   \centering
	\includegraphics[width=.4\linewidth]{./pictures/conv.jpg}
	\caption[Two steps of kernel convolution]{Two steps of kernel convolution}
      \label{fig:conv}
\end{figure}

With convolution, we reduce both computational requirements and a threat of overfitting by using local connections (representing weights) between input and output. However, these connections are local only in two dimensions, in width and height of the input; these connections have to be full along the channels depth - e.g. in RGB images, the last dimension of connections is always 3. Different is it with the output; its third dimension is determined by the number of neurons referencing the same spatial location, e.g. by the number of kernels we use. 

In chapter \ref{understanding-cnn}, translational invariance was mentioned. In convolutional layers, this invariance was achieved by another huge parameter reduction, by parameter sharing. Idea of parameter sharing raised from the premise that when one feature is useful in one location, it could be useful also in another one. This simple presumption which works except for for example centered special structures allows sharing a set of parameters throughout the whole depth slice. 

Using the parameters from \cite{conv-imagenet} as an example, assumpt that the feature map has size $55 \times 55 \times 96$ and we apply it on images of size $227 \times 227 \times 3$ using kernels of size $11 \times 11 \times 3$. Sharing parameters within a depth slice, we can reduce the parameter amount from $55 * 55 * 96 * (11 * 11 * 3 + 1)$ to $96 * (11 * 11 * 3 + 96)$ (1 and 96 are biases), it means from more than 105 millions to less than 35 thousands. Speaking only about the first layer. I believe that this example said it all. 

An inquiring reader may raise a question: In the chapter name, there is a plural. What happens in deeper layers? 

Their input is the previous layer output. Output of the first layer is the feature map of the lowest level features. As was already mentioned, each neuron of the next layer is connected with some local neighbourhood and with everything along the third dimension; and because the third dimension of the output is formed by stack of filters / kernels of the first layer, each second layer neuron is connected to all detected features in some location and its neighbourhood. The result? Output feature map from the second layer contains higher level features (simple combinations of the low level ones, like triangles or squares, simply combinations of some edges, curves, etc.). The next layer will output again higher level features and in the end, we may have very specific features like cars, reflective heliports or art deco swimming pools. 

\subsection{ReLU layers}
\label{relu-layers}



\subsection{Subsampling layers}
\label{subsampling}

\subsection{Normalization layers}
\label{norm-layers}

\subsection{Fully connected layers}
\label{fc-layers}

\section{Architectures of convolutional neural networks}
\label{cnn-architectures}

\subsection{LeNet5} %
\label{lenet}

\subsection{Dan Ciresan Net}
\label{ciresan}

\subsection{AlexNet} %
\label{alexnet}

% CS231n about zero-padding

\subsection{ZF Net}
\label{zfnet}

\subsection{VGG} %
\label{vgg}

\subsection{Network-in-network}
\label{nin}

\subsection{GoogLeNet}
\label{googlenet}

\subsection{Inception V3}
\label{inception}

\subsection{ResNet} %
\label{resnet}

\subsection{SqueezeNet}
\label{squeezenet}

\subsection{ENet}
\label{enet}
\chapter{CNNs for computer vision}
\label{image-ann}

A paper Visualizing and Understanding Convolutional Networks by Matthew D. Zeiler and Robert Fergus \cite{zf-net} started with two sentences: \textit{Large Convolutional Network models have recently demonstrated impressive classification performance on the ImageNet benchmark. However there is no clear understanding of why they perform so well, or how they might be improved.}

I tried to disperse such clouds a bit in the previous chapter, but now I would like to focus on another undertone connected with those statements. On their applications in the computer vision.

Almost everything mentioned in chapter \ref{cnn} was already tied to the computer vision. The following text will briefly describe the field of computer vision itself and then introduce few tasks in that field connected with the topic of the thesis. In each task, few \zk{CNN} models will be mentioned. This text structure was chosen also to depict the models evolution concluding in the one selected for the practical part of this thesis - an implementation into GRASS GIS.

% different sizes

\section{Understanding computer vision}
\label{computer-vision}

When you see a group photo, you can easily count the number of people in the photo, you can say whether they are smiling or not, whether they are happy, sad, angry, you can even guess whether they are one family, a bunch of friends, colleagues or just random people passing by. You can do all of that in a fraction of a second without any effort. The computer vision is supposed to be a computer-aided version of this human cognition. Or in a fancier way, from \cite{opencv}: \textit{Computer vision is the transformation of data from a still or video camera into either a decision or a new representation.} The new representation might be for instance a colour shift, the decision an answer on a question like \textit{Is there any football field in the picture?}

However, the problem is that the claim that it is easy for humans does not mean that it is easy also for computers. As is generally known and was indicated in chapter \ref{understanding-cnn}, the human brain is an extremely complex tool. The other thing is that a computer receives a visual impulse (image) in other way as is illustrated in figure \ref{fig:mirror}. Where human sees a side mirror, a computer sees a grid of numbers. And this grid may be completely different when the daytime, viewpoint, brightness, background or scale changes.

\begin{figure}[H]
   \centering
	\includegraphics[width=.8\linewidth]{./pictures/comp-vision.png}
	\caption[Human and computer cognition]{The difference between human and computer cognition, source: \cite{opencv}}
      \label{fig:mirror}
\end{figure}

The task may be something like \textit{Is there any side mirror in the picture?} This kind of tasks can be seen in the field of computer vision daily and can be extremely difficult to solve in the computer way. Although the input may include some metadata, it still has to be solved in a strict mathematical way. And because our goal is to make our computer vision system \textit{perceive like a human}, it looks like the right place for \zk{CNN}s.

Although we will focus only on classification and connected tasks like detection and segmentation, there are many more applications of the computer vision. To name a few: Autonomous cars, face recognition, fingerprint recognition, motion capture, biometrics and remote sensing.

To get a better view into the field of computer vision, it is recommended to read a richer source like \cite{comp-vision} and \cite{opencv} to get some practice.

\section{Classification}
\label{classification}

The idea is simple, image classification is the task of assigning an image to one class. It means that the user is interested in the result of a guess of what the image contains. A good example are camera traps. When the user is interested in mooses, he can train his model to predict mooses and automatically filter all outputs from the camera trap through this validator.

The idea of training a model to sorting a list of numbers is quite straightforward. Due to differences in pictures of the same object caused by influences mentioned in chapter \ref{computer-vision}, the classification task needs a bit of oblique, but still strictly logical thinking. It needs a data-driven approach. Instead of defining how exactly should moose look like, we feed the model with a bunch of labeled examples. It is the same process as with human children.

However, the number of classes does not have to be equal to one. The user can have a set of multple classes and even his classifier may differ. A popular simple classifier is a binary classifier returning just 1 or 0 (representing True or False/Yes or no) for each class per image, but much more widely used one is a softmax function giving a vector of probabilities for classes. This classifier also somehow more represents the human mind, as we may be unsure if there is a moose or a wapiti in the picture, if it was taken with bad conditions.

\begin{figure}[H]
   \centering
	\includegraphics[width=.8\linewidth]{./pictures/classification.png}
	\caption[Classification example]{An example of the classification output with softmax function, source: \cite{cnn-classification}}
      \label{fig:class}
\end{figure}

The pioneering work in the field of \zk{CNN}-based classification is AlexNet proposed in \cite{cnn-classification} and described in chapter \ref{alexnet}. Another interesting implementation is the CNN-RNN framework for multi-label image classification\footnote{Real world images often contain more features; multi-label image classification tries to predict more than just one of them. The problem of one-label classification can be seen in the seventh image in figure \ref{fig:class}.} proposed in \cite{multi-classification}.

\section{Classification with localization}
\label{classification-localization}

Although classification with localization is sometimes ignored in similar lists as a subtype of object detection, I believe it is useful to mention it as a step between pure classification and object detection.

Classification was already explained in the previous chapter. Classification with localization uses classification in its one-class form together with bounding boxes (bounding boxes may be multiple). The goal is to draw the bounding box as tight around the object as possible and predict the class of the object, so the user ends up with two outputs:
\begin{itemize}
	\item \textbf{Class:} Class label
	\item \textbf{Bounding box:} Box in the image defined by two coordinates and its width and height.
\end{itemize}

Because multiple values are returned, this task can be considered as a kind of a regression problem. Outputs from the regression are enough to get a result as the one illustrated in figure \ref{fig:class-loc}. 

\begin{figure}[H]
   \centering
	\includegraphics[width=.7\linewidth]{./pictures/class-loc.JPG}
	\caption[Classification with localization example]{An example of classification with localization}
      \label{fig:class-loc}
\end{figure}

\section{Object detection}
\label{object-detection}

Object detection is the classification with localization for multiple classes.

Shared basics with the task from the previous chapter instigate to use the similar approach, but applied on each single class individually. However, this method could be very, very inefficient. Different architectures use different practice to solve it and few of them will be presented here.

\subsection{R-CNN}
\label{r-cnn}

The Region-based convolutional neural network (\zk{R-CNN}) is a model proposed in 2014 in \cite{rcnn}, which combines region proposals with convolutional neural networks to detect objects in image via bounding boxes. 

The first step of the detection and also an answer to individual passes of classes is to generate category-independent region proposals containing probable objects. Instead of the whole image, those proposals are passed to a deep convolutional neural network which returns a feature vector for each region proposal. The last step is to pass this vector through a set of class-specific linear support vector machines (\zk{SVM}s).

Although diverse methods may be used for the region proposals generation\footnote{In the case of interest see an \textit{objectness measure} \cite{objectness} or \textit{category-independent object proposals with diverse ranking} \cite{cat-independent-proposals}.}, authors of the original R-CNN paper decided to use the selective search. The selective search was proposed in \cite{selective-search} and mixes advantages of both an exhaustive search and segmentation. From the exhaustive search, an effort to catch all possible object locations is used; from segmentation, the idea of following the image structure to guide the sampling process is used. To deal with many diverse image conditions, the selective search uses a variety of complementary image partitions.

After regions proposition, their features must be computed. Because the network works with a fixed-size input, the region must be warped into $227 \times 227$ square RGB image, and then it can be forward propagated through an architecture composed of five convolutional layers and two fully connected layers (a modified architecture of AlexNet, see chapter \ref{alexnet}).

The third step is scoring those extracted features (deciding which class the feature is and whether it is any class at all). This is done for each feature separately using a \zk{SVM} trained for that class.

In general, the bounding box has a large overlap with the background. This is the last problem with which the algorithm has to deal. To improve the localization (make bounding boxes tighter), a linear bounding box regression method proposed in \cite{object-det} is used with one modification - R-CNN applies regression on features computed by the CNN instead of on geometric features computed on the inferred deformable part model (\zk{DPM}) locations.

\begin{figure}[H]
   \centering
	\includegraphics[width=.9\linewidth]{./pictures/rcnn.png}
	\caption[R-CNN architecture]{R-CNN architecture schema, source: \cite{rcnn}}
      \label{fig:rcnn}
\end{figure}

Although R-CNN outperformed similar architectures\footnote{\cite{rcnn} claims a mean average precision (\zk{mAP}) of $53.3 \%$.}, there were still some shortcomings. The biggest one was the slowness caused mainly by three elements - the forward propagation through \zk{CNN} (each region of every image must be passed separately), by its triplicated training (a network for generating image features, a network for the class decision and the bounding box regression model) and by the generating of bounding box proposals.

\subsection{Fast R-CNN}
\label{fast-rcnn}

Then, in year 2015, a new architecture came to solve first two of these issues. Because the main reason for a new architecture was to speed-up \zk{R-CNN}, Ross Girshick named his new architecture proposed in \cite{fast-rcnn} simply Fast \zk{R-CNN}. Besides the main purpose to avoid first two issues mentioned in chapter \ref{r-cnn}, it also improves its accuracy\footnote{\cite{fast-rcnn} claims a \zk{mAP} of $66 \%$ on Pascal Visual object classes (\zk{VOC}) 2012. To find more info about the Pascal \zk{VOC} datasets and challenges, please see \cite{voc}.}.

The first issue, the separate forward propagations for all region proposals, was solved by propagating the entire image to obtain a feature map before the region proposition. For each object proposal is from the feature map extracted a fixed-length feature vector by a region of interest \zk{RoI} pooling layer.

The \zk{RoI} pooling layer can be seen as an one-level spatial pyramid pooling layer, a max pooling-based downsampling algorithm proposed in \cite{spp}. Its purpose is to decompose separately each valid \zk{RoI} into a fixed size $7 \times 7$ feature map. The decomposition is made by quantization each \zk{RoI} to the rounded discrete grid which is filled using max pooling on the corresponding kernel of the feature map.

These feature vectors are inputs for a set of \zk{FC} layers. This propagation is followed by the last, branched step, where two different outputs are obtained depending on the last layer - class probabilities from a softmax function and a bounding box defined by 4 values from a regressor.

There can be seen also a solution of the second problem named in chapter \ref{r-cnn}. Instead of three separate models, all steps are joint into only one model by appending classification (using a softmax layer instead of a separate \zk{SVM}) and bounding box regression as parallel layers to the end of the model.

\begin{figure}[H]
   \centering
	\includegraphics[width=.9\linewidth]{./pictures/fastrcnn.png}
	\caption[Fast R-CNN architecture]{Fast R-CNN architecture schema, source: \cite{fast-rcnn}}
      \label{fig:fast-rcnn}
\end{figure}

There can be raised a question whether is the \zk{SVM} replacement with the softmax layer as accurate as the original approach. According to tests performed in \cite{fast-rcnn}, the softmax layer is even slightly outperforming \zk{SVM} by 0.1 to 0.8 \zk{mAP} points depending on the depth of the network\footnote{Different models were tried. Suprisingly, with deeper network, smaller \zk{mAP} difference was noticed.}.

\subsection{Faster R-CNN}
\label{faster-rcnn}

The third speed issue mentioned in chapter \ref{r-cnn}, the region proposer based on the selective search, was solved in year 2016 in an architecture imaginatively named Faster \zk{R-CNN}, proposed in \cite{faster-rcnn}.

Microsoft Research team found that the feature map computed in the first part of Fast \zk{R-CNN} can be used to generate region proposals instead of using slower selective search algorithm. Authors did it by by including Region Proposal Network \zk{RPN} after the feature maps extraction of Fast R-CNN.

\begin{figure}[H]
   \centering
	\includegraphics[width=.5\linewidth]{./pictures/fasterrcnn-anchors.png}
	\caption[Region proposal network]{\zk{RPN} schema, source: \cite{faster-rcnn}}
      \label{fig:rpn}
\end{figure}

\zk{RPN} approach is different than the ones used in other architectures. Instead of pyramids of images or filters, \zk{RPN} uses anchor boxes - a set of rectangular bounding boxes proposals and scores created by sliding a spatial window over the entire feature map. The sliding window is a $n \times n$ fully convolutional network. The anchor boxes are defined by a scale and aspect ratio, so they are of different shapes. Powerful attributes of anchor boxes are that they are translation-invariant and multi-scale, which concludes also into a reduction of the model size.

\begin{figure}[H]
   \centering
	\includegraphics[width=.55\linewidth]{./pictures/fasterrcnn.png}
	\caption[Faster R-CNN architecture]{Faster \zk{R-CNN} architecture schema, source: \cite{faster-rcnn}}
      \label{fig:faster-rcnn}
\end{figure}

As can be seen in figure \ref{fig:faster-rcnn}, the training scheme alternates between training (or fine-tuning) for the region proposition and for object detection. This approach with shared convolutional features quickly converges.

Besides the speed improvement, Faster \zk{R-CNN} improves also the accuracy. \cite{faster-rcnn} claims a \zk{mAP} of $73.2 \%$ on Pascal \zk{VOC} 2007 and of $70.4 \%$ on Pascal \zk{VOC} 2012.

% Single-Shot MultiBox Detector (SSD) You Only Look Once (YOLO)

\section{Semantic segmentation}
\label{semantic-segmentation}

% CNN

\section{Instance segmentation}
\label{instance-segmentation}

% Mask R-CNN

% \section{Content-based Image Retrieval}
\chapter{Used technologies}
\label{technologies}

This chapter briefly introduces the most important technologies used during
the~development of modules for GRASS GIS. It means that besides GRASS GIS itself, 
the~Python language and its libraries TensorFlow and Keras will be introduced.

\section{GRASS GIS}
\label{grass}

\begin{figure}[H]
   \centering
	\includegraphics[width=0.4\linewidth]{./pictures/grass-logo.png}
	\caption[GRASS GIS logo]{GRASS GIS logo, source: \url{https://grass.osgeo.org/download/logos/}}
      \label{fig:grass-logo}
\end{figure}

The history of \zk{GRASS} \zk{GIS} (an acronym for Geographic Resources Analysis 
Support System Geographic Information System) started in year 1982 at the U.S. 
Army Corps of Engineers Construction Engineering Research Laboratory, but the 
version 1.0 under the name \zk{GRASS} was released later, in 1985. However, its 
way lead from the national project to the international Open GRASS Consortium in 
the mid 1990s. This consortium is perceived as an ancestor of today's Open 
Geospatial Consortium (\zk{OGC}). In 1999, \zk{GRASS} \zk{GIS} was published 
under GNU General Public License (\zk{GPL}).

\zk{GRASS} \zk{GIS} grown to a cross-platform free and open-source \zk{GIS} 
maintained by an international team of developers and users and licensed under 
the GNU \zk{GPL} license allowing users to perform geospatial data management 
and analysis for both raster and vector data, image processing, geocoding and 
visualization. Within its more than 300 modules, it supports also 
spatio-temporal data.

For more info about \zk{GRASS} \zk{GIS}, please see its original 
website\footnote{\url{https://grass.osgeo.org/documentation/general-overview/}}.

\section{Python}
\label{python}

\begin{figure}[H]
   \centering
	\includegraphics[width=\linewidth]{./pictures/python-logo.png}
	\caption[Python logo]{Python logo, source: \url{https://www.python.org/community/logos/}}
      \label{fig:python-logo}
\end{figure}

Python is a cross-platform open-source interpreted programming language. Python 
was invented by a~Dutch programmer and Monty Python's Flying Circus fan Guido 
van~Rossum in year 1991. Due to its user-friendly and readable structure, it is 
more and more popular.

It is very popular also in the field of open-source \zk{GIS}. Many modules for 
\zk{GRASS} \zk{GIS} are written in Python and Python is the most used language 
in QGIS\footnote{\url{https://qgis.org/}}.

For info about programming in Python, please see \cite{diveintopython}.

\section{TensorFlow}
\label{tf}

\begin{figure}[H]
   \centering
	\includegraphics[scale=0.5]{./pictures/tf-logo.png}
	\caption[TensorFlow logo]{TensorFlow logo, source: \url{https://www.tensorflow.org/images/tf_logo_transp.png}}
      \label{fig:tf-logo}
\end{figure}

TensorFlow is an open-source software library developed by Google Brain 
Team\footnote{\url{https://research.google.com/teams/brain/}} primarily for the 
purpose of neural network research. TesorFlow uses data flow graphs, where a 
node represent a mathematical operation and an edge represents a~tensor (a 
multidimensional array). A toolkit for visualization of these graphs is called 
TensorBoard.

TensorFlow is available for Python and C++ programming languages. Its usage 
optimizes mathematical expressions and as it was developed for the purpose of 
neural networks, the main focus was dedicated to this field and TensorFlow 
provides a lot of functions useful for deep learning.

A good starting point is book Getting Started with TensorFlow \cite{tf}. A 
documentation and examples can be found also at the original 
website\footnote{\url{https://www.tensorflow.org/}}.

\section{Keras}
\label{keras}

\begin{figure}[H]
   \centering
	\includegraphics[width=\linewidth]{./pictures/keras-logo.png}
	\caption[Keras logo]{Keras logo, source: \url{https://keras.io/}}
      \label{fig:keras-logo}
\end{figure}

Keras is an open-source software library written in Python and using TensorFlow, 
Microsoft Cognitive Toolkit, Theano or MXNet as a backend tensor manipulation 
library.

Keras is again developed for the purpose of deep learning. Its powerful feature 
is the content of extensible objects for defining different types of layers (see 
chapter \ref{layers}), models, loss functions and other tools widely used in the 
field of \zk{ANN}s.

For a wider documentation, please see the original 
website\footnote{\url{https://keras.io/}}.

\chapter{Implementation}
\label{implementation}

Mask \zk{R-CNN} tools created for the practical part of the thesis consist of two modules. \verb|ann.maskrcnn.train| allows user to train a Mask R-CNN model on his own dataset, \verb|ann.maskrcnn.detect| allows him to use that model to detect features in georeferenced files. 

Along with these modules, a library of Mask \zk{R-CNN} tools was prepared. This library was heavily based on a Python implementation of Mask \zk{R-CNN} written by Waleed Abdulla from Matterport, Inc.\footnote{\url{https://matterport.com/about/}}  Matterport, Inc. published their implementation under the MIT License \cite{mit}. The MIT License is a license granting the permission to use the code, copy it, modify it, publish and even to sell it free of charge and is compatible with GNU General Public License 2 (or newer) \cite{gplv2} of GRASS GIS. Scripts in the library based on Abdulla's code are also under the MIT License and moreover, Waleed Abdulla himself agreed with the usage and modifications of his code for purposes of the GRASS GIS usage. The Matterport, Inc. Mask \zk{R-CNN} implementation can be found in their github repository\footnote{\url{https://github.com/matterport/Mask\_RCNN}}.

% TODO: Tensorflow and Keras usage

The following text will briefly describe the structure of the mentioned modules together with their workflow. Because aspects of the Mask \zk{R-CNN} architecture were already mentioned in chapter \ref{mask-rcnn}, this facet will be a bit overshadowed and the main focus will be given to the code implementation. The library will be also introduced altogether with notes on my modifications connected with this thesis to distinguish them from Abdulla's code.

\section{Mask R-CNN library}
\label{library}

The library consists of five files:
\begin{itemize}
	\item \verb|config.py|: The configuration file for the model. 
	 \item \verb|model.py|: The core of the Mask \zk{R-CNN} model. Builds up the model.
	 \item \verb|parallel_model.py|: Contains the ParallelModel class, a subclass of the standard Keras model allowing the parallelized computation. This file is in the original state written by Waleed Abdulla without any modification.
	 \item \verb|utils.py|: Utilities for the model. Utilities are box overlaps computation, bounding box computation from detected masks and their refinements, image resizing, pyramid anchors and most importantly the Dataset class loading and parsing images in the dataset.
	 \item \verb|visualize.py|: The file providing visualization tools for the detection. The most important part is saving detected objects as raster files.
\end{itemize}

\subsection{Configuration file}
\label{config}



\section{ann.maskrcnn.train}
\label{train-module}

\section{ann.maskrcnn.detect}
\label{detect-module}

\section{License}
\label{license}
\chapter{Conclusion}
\label{conclusion}

The goal of the~thesis was to find and implement a~suitable \zk{ANN} 
architecture into \zk{GRASS} \zk{GIS}. Because \zk{ANN}s and especially 
\zk{CNN}s are shaking with the~field of computer vision, many GRASS GIS users 
and developers were interested in implementation options into this system.

The first part of the~thesis was dedicated to a~theoretical background behind 
\zk{CNN}s. It is followed by their various applications in the~field of computer 
vision. 

The second part of the~thesis was dedicated to the implementation of Mask
\zk{R-CNN} modules into GRASS GIS. It starts with a~brief introduction of some
of the~used tools and follows with explanations of the~most important parts of
the~code.

The research flows in the~background of the~theoretical part and is briefly 
summarized at the~beginning of the~chapter on the~implementation.

Developed modules are ready to use and as can be seen in appendices, I already 
made few tests. However, because the~training costs a~lot of time (at least few 
days on a~GPU machine, even weeks on a~machine without GPU), only a~limited 
amount of tests was made. Modules were proposed for testing also to other GRASS 
GIS developers and users and they found their results to be satisfying, 
sometimes even for data, where other methods of classification in GRASS GIS 
failed (a shadow over a~field, different colours, etc.).

Developed modules are available from the GitHub
repository\footnote{\url{https://github.com/ctu-geoforall-lab-projects/dp-pesek-2018}}.

Even though modules work, there are still some possible extensions and even
some issues.

The biggest issue is in insufficient support for Python 3 in GRASS GIS. This 
issue is solved by a~patch attached as an e-attachment and should be solved in 
GRASS GIS during this year.

Possible extensions are:
\begin{itemize}
	\item Support more training data annotation structures. The~current state
	works with \verb|*.jpg| images and \verb|*.png| masks for each instance.
	The~next step could be a~JSON-based structure used in MS
	COCO\footnote{\url{http://cocodataset.org/\#format-data}}.
	\item Support training on images in GRASS GIS (using raster map as images,
	vector areas as masks).
	\item Support more channels than three.
	\item Support more types of the~backbone architecture. Now only \zk{ResNet} 50
	and \zk{ResNet} 101 are supported.
	\item Diversify head architecture types. The~current one is just very basic
	one.
	\item Implement image augmentation for the training.
\end{itemize}

The last thing worthy of mention is the~high gear of progress. After only six 
years of research in the~field of \zk{CNN}s, black is white and up is down. 
Everything has changed and results unbelievable six years back are standards 
nowadays. Almost every week, there is a~new architecture. Even during work on 
this project, many interesting architectures were proposed, to name a~few, 
\cite{masklab} or \cite{panoptic}.

Developed modules are the~first step of \zk{GRASS} \zk{GIS} in the~direction to 
neural networks. Many functions could be useful for developing more modules. It 
would be pleasant to start a~new trend and see more modules using \zk{CNN}s and 
sharing libraries. Unsupervised learning is an open call.


% vysázení seznamu zkratek

\begin{seznamzkratek}{ABCDE}

	\novazkratka{AI}
	      {AI}
	      {Artificial intelligence}
	\novazkratka{ANN}
	      {ANN}
	      {Artificial neural network}
	\novazkratka{CNN}
	      {CNN}
	      {Convolutional neural network}
	\novazkratka{DPM}
	      {DPM}
	      {Deformable part model}
	\novazkratka{FAIR}
	      {FAIR}
	      {Facebook AI research}
	\novazkratka{FC}
	      {FC}
	      {Fully connected}
	\novazkratka{FCN}
	      {FCN}
	      {Fully convolutional network}
	\novazkratka{FPN}
	      {FPN}
	      {Feature pyramid network}
	\novazkratka{GIS}
	      {GIS}
	      {Geographic information system}
	\novazkratka{GRASS}
	      {GRASS}
	      {Geographic resources analysis support system}
	\novazkratka{GUI}
	      {GUI}
	      {Graphical user interface}
	\novazkratka{ILSVRC}
	      {ILSVRC}
	      {ImageNet large scale visual recognition challenge}
	\novazkratka{IoU}
	      {IoU}
	      {Intersection over union}
	\novazkratka{mAP}
	      {mAP}
	      {Mean average precision}
	\novazkratka{R-CNN}
	      {R-CNN}
	      {Region-based convolutional neural network}
	\novazkratka{ReLU}
	      {ReLU}
	      {Rectified linear unit}
	\novazkratka{ResNet}
	      {ResNet}
	      {Residual network}
	\novazkratka{RNN}
	      {RNN}
	      {Recurrent neural network}
	\novazkratka{RoI}
	      {RoI}
	      {Region of interest}
	\novazkratka{RPN}
	      {RPN}
	      {Region proposal network}
	\novazkratka{SVM}
	      {SVM}
	      {Support vector machine}
	\novazkratka{VOC}
	      {VOC}
	      {Visual object classes}
	      
\end{seznamzkratek}

% literatura
\nocite{*}
\def\refname{References}
\bibliographystyle{mystyle}
\bibliography{literatura}


% začátek příloh
%\def\figurename{Figure}%
\prilohy

% vysázení seznamu příloh
\seznampriloh

% Vložení souboru s přílohami
\chapter{User manual}
\label{manual}

HTML pages in the~style of common GRASS GIS user manuals were created. In this
chapter, they will be attached. Firstly, they will introduce Mask R-CNN tools
generally and then each module separately.

\section{Mask R-CNN tools}
\label{appendix-lib}

\subsection*{DESCRIPTION}

Mask R-CNN tools allow the~user to train his own model and use it for
a~detection of objects, or to use a~model provided by someone else. It can be
seen as a~supervised classification using convolutional neural networks.

The training is done using module \emph{i.ann.maskrcnn.train}. The~user feeds
the~module with training data consisting of images and masks for each instance
of intended classes and gets a~model. For difficult tasks and when not using
a~pretrained model, the~training may take even weeks; in case of a~good
pretrained model and powerful PC with GPU support, the~training could get good
results after 1 day and even less.

When the~user has a~trained model, it can be used for the~detection. Module
\emph{i.ann.maskrcnn.detect} detects classes learned during the~training and 
extracts from given images vectors corresponding to detected objects. Objects 
can be extracted either as areas or points. 

\subsection*{DEPENDENCIES}

\emph{i.ann.maskrcnn.*} modules contain a~lot of external python dependencies.
To run modules, it is necessary to have them installed. Modules use Python 3,
so please install Python 3 versions.

\liststyleLi
\begin{itemize}
\item NumPy
\item Pillow
\item SciPy
\item Cython
\item scikit-image
\item OSGeo
\item TensorFlow
\item Keras
\item h5py
\end{itemize}

After dependencies are fulfilled, modules can be installed locally using
\emph{g.extension} module:
{\footnotesize
\begin{lstlisting}[breaklines=true]
g.extension extension=i.ann.maskrcnn url=path/to/the/i.ann.maskrcnn/folder
\end{lstlisting}
}

After submitting the thesis, modules will be added to GRASS GIS official
AddOns. It should be possible to install them directly by this command:
{\footnotesize
\begin{lstlisting}
g.extension extension=i.ann.maskrcnn
\end{lstlisting}
}

\subsection*{GRASS GIS PATCH}

Unfortunately, python3 is not fully supported by GRASS GIS yet. To use
environment setting flags like \emph{--overwrite}, the~user has to update
his GRASS GIS with the~following patch:

{\footnotesize
\begin{lstlisting}[breaklines=true]
===================================================================
--- lib/python/script/core.py	(revision 72644)
+++ lib/python/script/core.py	(working copy)
@@ -746,7 +746,7 @@
         elif var.startswith(b'opt_'):
             options[var[4:]] = val
         elif var in [b'GRASS_OVERWRITE', b'GRASS_VERBOSE']:
-            os.environ[var] = val
+            os.environ[var.decode("utf-8")] = val.decode("utf-8")
         else:
             raise SyntaxError("invalid output from g.parser: %s" % line)
\end{lstlisting}
}

\clearpage

\section{i.ann.maskrcnn.train}
\label{appendix-train}

\subsection*{NAME}

\textbf{i.ann.maskrcnn.train -} Train your Mask R-CNN network.

\subsection*{SYNOPSIS}

\begin{flushleft}
\textbf{i.ann.maskrcnn.train} 

\textbf{i.ann.maskrcnn.train -{}-help}

\textbf{i.ann.maskrcnn.train} [\textbf{-esbn}] \textbf{training\_dataset}=name
[\textbf{model}=string] \tab\ \textbf{classes}=string[,string,...]
\textbf{logs}=name \textbf{name}=string [\textbf{epochs}=value] \tab\
[\textbf{steps\_per\_epoch}=value] [\textbf{rois\_per\_image}=value] \tab\
[\textbf{images\_per\_gpu}=value] [\textbf{gpu\_count}=value] \tab\
[\textbf{mini\_mask\_size}=value[,value,...]]
[\textbf{validation\_steps}=value] \tab\ [\textbf{images\_min\_dim}=value]
[\textbf{images\_max\_dim}=value] \tab\ [\textbf{backbone}=string]
[\textbf{-{}-overwrite}] [\textbf{-{}-help}] [\textbf{-{}-verbose}]
[\textbf{-{}-quiet}] [\textbf{-{}-ui}]
\end{flushleft}

\subsubsection*{Flags:}
\begin{flushleft}
  \textbf{-e}
  
  \tab Pretrained weights were trained on another classes / resolution / sizes
  
  \textbf{-s}
  
  \tab Do not use 10 \% of images and save their list to logs dir
  
  \textbf{-b}
  
  \tab Train also batch normalization layers (not recommended for small
  batches)

  \textbf{-n}
  
  \tab No resizing or padding of images (images must be of the~same size)
  
  \textbf{-{}-overwrite}
  
  \tab Allow output files to overwrite existing files
  
  \textbf{-{}-help}
  
  \tab Print usage summary
  
  \textbf{-{}-verbose}
  
  \tab Verbose module output
  
  \textbf{-{}-quiet}
  
  \tab Quiet module output
  
  \textbf{-{}-ui}
  
  \tab Force launching GUI dialog
\end{flushleft}

\subsubsection*{Parameters:}

\begin{flushleft}
\textbf{training\_dataset}=name \textbf{[required]}

\tab Path to the~dataset with images and masks

\textbf{model}=string

\tab Path to the~.h5 file to use as initial values

\tab Keep empty to train from a~scratch

\textbf{classes}=string[,string,...] \textbf{[required]}
           
\tab Names of classes separated with ","

\textbf{logs}=name \textbf{[required]}

\tab Path to the~directory in which will be models saved

\textbf{name}=string \textbf{[required]}

\tab Name for output models

\textbf{epochs}=value

\tab Number of epochs

\tab default: 200

\textbf{steps\_per\_epoch}=value

\tab Steps per each epoch

\tab default: 3000

\textbf{rois\_per\_image}=value

\tab How many ROIs train per image

\tab default: 64

\textbf{images\_per\_gpu}=value

\tab Number of images per GPU

\tab Bigger number means faster training but needs a~bigger GPU

\tab default: 1

\textbf{gpu\_count}=value

\tab Number of GPUs to be used

\tab default: 1

\textbf{mini\_mask\_size}=value[,value,...]

\tab Size of mini mask separated with ","

\tab To use full sized masks, keep empty.

\tab Mini mask saves memory at the~expense of precision

\textbf{validation\_steps}=value

\tab Number of validation steps

\tab Bigger number means more accurate estimation of the~model precision

\tab default: 100

\textbf{images\_min\_dim}=value

\tab Minimum length of images sides

\tab Images will be resized to have their shortest side at least of this value

\tab Has to be a~multiple of 64

\tab default: 256

\textbf{images\_max\_dim}=value

\tab Maximum length of images sides

\tab Images will be resized to have their longest side of this value

\tab Has to be a~multiple of 64

\tab default: 1280

\textbf{backbone}=string

\tab Backbone architecture

\tab options: resnet50, resnet101

\tab default: resnet101
\end{flushleft}

\subsection*{DESCRIPTION}
\textstyleEmphasis{i.ann.maskrcnn.train} allows the user to train a~Mask R-CNN
model on his own dataset. The~dataset has to be prepared in a~predefined
structure. 

\subsubsection*{DATASET STRUCTURE}
Training dataset should be in the~following structure: 

\liststyleLi
\begin{itemize}
\item dataset-directory
\begin{itemize}
\item imagenumber 

\begin{itemize}
\item imagenumber.jpg (training image) 
\item imagenumber-class1-number.png (mask for one instance of class1) 
\item imagenumber-class1-number.png (mask for another instance of class1) 
\item ... 
\end{itemize}
\item imagenumber2 

\begin{itemize}
\item imagenumber2.jpg 
\item imagenumber2-class1-number.png (mask for one instance of class1) 
\item imagenumber2-class2-number.png (mask for another class instance) 
\item ... 
\end{itemize}
\end{itemize}
\end{itemize}

The described structure of directories is required. Pictures must be *.jpg
files with 3 channels (for example RGB), masks must be *.png files consisting
of numbers between 1 and 255 (object instance) and 0s (elsewhere). A~mask file
for each instance of an object should be provided separately distinguished by
the suffix number. 

\subsection*{NOTES}
If you are using initial weights (the \textstyleEmphasis{model} parameter),
epochs are divided into three segments. Firstly training layers 5+, then
fine-tuning layers 4+ and the~last segment is fine-tuning the~whole
architecture. Ending number of epochs is shown for your segment, not for the
whole training. 

The usage of the~\textstyleEmphasis{{}-b} flag will result in an activation of
batch normalization layers training. By default, this option is set to False,
as it is not recommended to train them when using just small batches (batch is
defined by the~\textstyleEmphasis{images\_per\_gpu} parameter). 

If the~dataset consists of images of the~same size, the~user may use the
\textstyleEmphasis{{}-n} flag to avoid resizing or padding of images. When the
flag is not used, images are resized to have their longer side equal to the
value of the~\textstyleEmphasis{images\_max\_dim} parameter and the~shorter
side longer or equal to the~value of the~\textstyleEmphasis{images\_min\_dim}
parameter and zero-padded to be of shape \verb|images_max_dim| $\times$
\verb|images_max_dim|. It results in the~fact that even images of different
sizes may be used. 

After each epoch, the~current model is saved. It allows the~user to stop
the~training when he feels satisfied with loss functions. It also allows
the~user to test models even during the~training (and, again, stop it even
before the~last epoch). 

\subsection*{EXAMPLES}
Dataset for examples: 

\liststyleLii
\begin{itemize}
\item crops 

\begin{itemize}
\item 000000 

\begin{itemize}
\item 000000.jpg 
\item 000000-corn-0.png 
\item 000000-corn-1.png 
\item ... 
\end{itemize}
\item 000001 

\begin{itemize}
\item 000001.jpg 
\item 000001-corn-0.png 
\item 000001-rice-0.png 
\item ... 
\end{itemize}
\end{itemize}
\end{itemize}

\subsubsection*{Training from scratch}

{\footnotesize
\begin{lstlisting}[breaklines=true]
i.ann.maskrcnn.train training_dataset=/home/user/Documents/crops classes=corn,rice logs=/home/user/Documents/logs name=crops
\end{lstlisting}
}

After default number of epochs, we will get a~model where the~first class is
trained to detect corn fields and the~second one to detect rice fields. 

If we use the~command with reversed classes order, we will get a~model where
the first class is trained to detect rice fields and the~second one to detect
corn fields:

{\footnotesize
\begin{lstlisting}[breaklines=true]
i.ann.maskrcnn.train training_dataset=/home/user/Documents/crops classes=rice,corn logs=/home/user/Documents/logs name=crops
\end{lstlisting}
}

The name of the~model does not have to be the~same as the~dataset folder but
should be referring to the~task of the~dataset. A~good name for this one
(referring also to the~order of classes) could be also this one: 

{\footnotesize
\begin{lstlisting}[breaklines=true]
i.ann.maskrcnn.train training_dataset=/home/user/Documents/crops classes=rice,corn logs=/home/user/Documents/logs name=rice_corn
\end{lstlisting}
}

\subsubsection*{Training from a~pretrained model}
We can use a~pretrained model to make our training faster. It is necessary for
the~model to be trained on the~same channels and similar features, but it does
not have to be the~same ones (e.g. model trained on swimming pools in maps can
be used for a~training on buildings in maps). 

A model trained on different classes (use \textstyleEmphasis{{}-e} flag to
exclude head weights):

{\footnotesize
\begin{lstlisting}[breaklines=true]
i.ann.maskrcnn.train training_dataset=/home/user/Documents/crops classes=corn,rice logs=/home/user/Documents/logs name=crops model=/home/user/Documents/models/buildings.h5 -e
\end{lstlisting}
}

A model trained on the~same classes:

{\footnotesize
\begin{lstlisting}[breaklines=true]
i.ann.maskrcnn.train training_dataset=/home/user/Documents/crops classes=corn,rice logs=/home/user/Documents/logs name=crops model=/home/user/Documents/models/corn_rice.h5
\end{lstlisting}
}

\subsubsection*{Fine-tuning a~model}
It is also possible to stop your training and then continue. To continue in the
training, just use the~last saved epoch as a~pretrained model:

{\footnotesize
\begin{lstlisting}[breaklines=true]
i.ann.maskrcnn.train training_dataset=/home/user/Documents/crops classes=corn,rice logs=/home/user/Documents/logs name=crops model=/home/user/Documents/models/mask_rcnn_crops_0005.h5
\end{lstlisting}
}

\clearpage

\section{i.ann.maskrcnn.detect}
\label{appendix-detect}

\subsection*{NAME}

\textbf{i.ann.maskrcnn.detect -} Detect features in images using a~Mask R-CNN
model.

\subsection*{SYNOPSIS}

\begin{flushleft}
\textbf{i.ann.maskrcnn.detect} 

\textbf{i.ann.maskrcnn.detect -{}-help}

\textbf{i.ann.maskrcnn.detect} [\textbf{-esbn}] \textbf{images\_directory}=name
\tab\ \textbf{images\_format}=string \textbf{model}=string
\textbf{classes}=string[,string,...] \tab\ [\textbf{masks\_output}=name]
[\textbf{output\_type}=string] [\textbf{-{}-overwrite}] [\textbf{-{}-help}]
\tab\ [\textbf{-{}-verbose}] [\textbf{-{}-quiet}] [\textbf{-{}-ui}]
\end{flushleft}

\subsubsection*{Flags:}
\begin{flushleft}
  \textbf{-e}
  
  \tab External georeferencing in the~images folder
  
  \textbf{-{}-overwrite}
  
  \tab Allow output files to overwrite existing files
  
  \textbf{-{}-help}
  
  \tab Print usage summary
  
  \textbf{-{}-verbose}
  
  \tab Verbose module output
  
  \textbf{-{}-quiet}
  
  \tab Quiet module output
  
  \textbf{-{}-ui}
  
  \tab Force launching GUI dialog
\end{flushleft}

\subsubsection*{Parameters:}

\begin{flushleft}
\textbf{band1}=name

\tab Name of raster maps to use for detection as the first band
(divided by ",")

\textbf{band2}=name

\tab Name of raster maps to use for detection as the second band
(divided by ",")

\textbf{band3}=name

\tab Name of raster maps to use for detection as the third band
(divided by ",")

\textbf{images\_directory}=name

\tab Path to a~directory with images to detect

\textbf{images\_format}=string

\tab Format suffix of images

\textbf{model}=string \textbf{[required]}

\tab Path to the~.h5 file containing the~model

\textbf{classes}=string[,string,...] \textbf{[required]}
           
\tab Names of classes separated with ","

\textbf{masks\_output}=name

\tab Directory where masks will be saved

\textbf{output\_type}=string

\tab Type of output

\tab options:area, point

\tab default: area
\end{flushleft}

\subsection*{DESCRIPTION}
\textstyleEmphasis{i.ann.maskrcnn.detect} allows the user to use a~Mask R-CNN
model to detect features in GRASS GIS raster maps or georeferenced files and
extract them either as areas or
points. The~module creates a~separate map for each class. 

\subsection*{NOTES}

The detection may be used for raster maps imported in GRASS GIS or for external
files (or using both). To use raster maps in GRASS GIS, you need to pass them
in three bands following the order used during the training, e.g. if the
training has been made on RGB images, use \textstyleEmphasis{band1=*.red</em},
\textstyleEmphasis{band1=*.green} and \textstyleEmphasis{band3=*.blue}. To
pass multiple images, put more maps into \textstyleEmphasis{band*} parameters,
divided by ",".

The detection may be used also for multiple external files. However, all files
for the detection must be in one directory specified in the
\textstyleEmphasis{images\_directory} parameter. Even when using only one
image, the~module finds it through this parameter. 

When detecting, you can use new names of classes. Classes in the~model are not
referenced by their name, but by their order. It means that if the~model was
trained with classes \textstyleEmphasis{corn,rice} and you use
\textstyleEmphasis{i.ann.maskrcnn.detect} with classes
\textstyleEmphasis{zea,oryza}, zea areas will present areas detected as corn
and oryza areas will present areas detected as rice. 

If the~external file is georeferenced externally (by a~worldfile or an
\textstyleEmphasis{.aux.xml} file), please use \textstyleEmphasis{{}-e} flag. 

\subsection*{EXAMPLES}
\subsubsection*{Detect buildings and lakes and import them as areas}

One map imported in GRASS GIS:

{\footnotesize
\begin{lstlisting}[breaklines=true]
i.ann.maskrcnn.detect band1=map1.red band2=map1.green band3=map1.blue classes=buildings,lakes model=/home/user/Documents/logs/mask_rcnn_buildings_lakes_0100.h5
\end{lstlisting}
}

\ \linebreak Two maps (map1, map2) imported in GRASS GIS:

{\footnotesize
\begin{lstlisting}[breaklines=true]
i.ann.maskrcnn.detect band1=map1.red,map2.red band2=map1.green,map2.green band3=map1.blue,map2.blue classes=buildings,lakes model=/home/user/Documents/logs/mask_rcnn_buildings_lakes_0100.h5
\end{lstlisting}
}

\ \linebreak External files, the~georeferencing is internal (GeoTIFF): 

{\footnotesize
\begin{lstlisting}[breaklines=true]
i.ann.maskrcnn.detect images_directory=/home/user/Documents/georeferenced_images classes=buildings,lakes model=/home/user/Documents/logs/mask_rcnn_buildings_lakes_0100.h5 images_format=tif
\end{lstlisting}
}

\ \linebreak External files, the~georeferencing is external: 

{\footnotesize
\begin{lstlisting}[breaklines=true]
i.ann.maskrcnn.detect images_directory=/home/user/Documents/georeferenced_images classes=buildings,lakes model=/home/user/Documents/logs/mask_rcnn_buildings_lakes_0100.h5 images_format=png  -e
\end{lstlisting}
}

\subsubsection*{Detect cottages and plattenbaus and import them as points}

{\footnotesize
\begin{lstlisting}[breaklines=true]
i.ann.maskrcnn.detect band1=map1.red band2=map1.green band3=map1.blue classes=buildings,lakes model=/home/user/Documents/logs/mask_rcnn_buildings_lakes_0100.h5 output_type=point
\end{lstlisting}
}

\chapter{Examples}
\label{examples}

In the~following chapter, few results obtained during tests of modules will be
presented. Firstly a~module trained to detect football and tennis pitches, then 
a module trained to detect buildings.

\section{Pitches}

The model for detecting football and tennis pitches was trained on a~Debian 
server with 16 CPUs Intel Xeon E5540 (8 CPUs were in the~use) and with memory
49~GBs. The~processor base frequency of Intel Xeon E5540 is 2.53 GHz and
the~cache is 8~MB. The~training used a~model trained on the~MS COCO dataset as
a~pre-trained model and the~training took one month reaching loss function of
0.9568.

The training dataset consisted of Bing 
maps\footnote{\url{https://www.bing.com/maps}} tiles with zoom level 18 
(resolution 0.6 m per pixel) and masks corresponding to above-mentioned
pitches. The training dataset consisted of almost 54000 images (plus masks).

When the~training was stopped, the~loss function was about 0.86.

Detection of pitches worked very well on images with the same shape and
resolution (0.6 m/pixel) as training images (figures \ref{fig:football1} and
\ref{fig:tennis1}). For images with worse resolution (1.19 m/pixel), the
detection sometimes contained a bit of background for tennis pitches (figures
\ref{fig:football-tennis}). However, there
was a problem in detecting football pitches, as the model sometimes detected
wrongly also other green fields (figure \ref{fig:football2}). This problem is
considered to be caused by the~fact that training data contained also amateur
football pitches consisting only of a rectangular green field and two goals.

\begin{figure}[H]
   \centering
	\includegraphics[width=.9\linewidth]{./pictures/out1.png}
	\caption[Detection of football pitches]{An example of the~detection on
	a~picture containing football pitches}
      \label{fig:football1}
\end{figure}

\begin{figure}[H]
   \centering
	\includegraphics[width=.9\linewidth]{./pictures/out2.png}
	\caption[Detection of tennis pitches]{An example of the~detection on
	a~picture containing tennis pitches}
      \label{fig:tennis1}
\end{figure}

\begin{figure}[H]
   \centering
	\includegraphics[width=.9\linewidth]{./pictures/out4.png}
	\caption[Detection of football pitches, another resolution]{An example of
	the~detection on a~picture containing football pitches}
      \label{fig:football2}
\end{figure}

\begin{figure}[H]
   \centering
	\includegraphics[width=.9\linewidth]{./pictures/out3.png}
	\caption[Detection of football and tennis pitches]{An example of the~detection
	on a~picture containing football and tennis pitches}
      \label{fig:football-tennis}
\end{figure}

The permission to the usage of Bing map tiles was given to me specifically for 
purposes of this thesis. Unfortunately, the~permission to share the~training 
dataset in e-attachments could not be given to me.

\section{Buildings}

Another model was trained to recognize buildings. Training data were from the
same source, with the same resolution, but this time only 2400 images (plus
masks) was used.

The training ran for 2 days on a machine using NVIDIA Tesla K80 GPU. NVIDIA
Tesla K80 has memory 24 GBs, effective clock speed 2.5 Hz with 875 MHz boost
clock and 560 MHz core clock. The training ran from the scratch.

Results are worse than in the pitches detection. In the figure from the last
epoch, figure \ref{fig:build-1},
we can see a~tennis pitch and road detected as a building, the building with
a black roof was not recognized and only a small part of the~tribune was
detected.
These problems may be caused by some of the following effects:
\begin{itemize}
	\item More than twenty times smaller dataset.
	\item Bigger diversity in building types (compared to tennis pitches)
	\item Training from a scratch and for a shorter time
\end{itemize}

Even though the training took a shorter time, it reached much smaller loss
function (0.5019 for the last epoch) than in the pitches training. A figure of
detection with this loss
function is shown in figure \ref{fig:build-1}.

To illustrate
a~progress of the~training, the~same area will be
shown also during older epochs with loss function depicted in their captions.
The progress shows also interesting moment, where an epoch with loss function
0.6280 (figure \ref{fig:build-6}) seems to behave better than the one with loss
function 0.5019 (it detects also the small building on the left and a piece of
a building at the bottom of the~picture and it does not detect the street on
the right; the tribune was detected as two individual buildings).

\begin{figure}[H]
   \centering
	\includegraphics[width=.9\linewidth]{./pictures/out_b_1.png}
	\caption[Detection of buildings, example]{An example of the~detection; epoch
	1, loss function 35.0102}
      \label{fig:build-3}
\end{figure}

\begin{figure}[H]
   \centering
	\includegraphics[width=.9\linewidth]{./pictures/out_b_10.png}
	\caption[Detection of buildings, example]{An example of the~detection; epoch
	10, loss function 5.8694}
      \label{fig:build-2}
\end{figure}

\begin{figure}[H]
   \centering
	\includegraphics[width=.9\linewidth]{./pictures/out_b_50.png}
	\caption[Detection of buildings, example]{An example of the~detection; epoch
	50, loss function 1.3638}
      \label{fig:build-4}
\end{figure}

\begin{figure}[H]
   \centering
	\includegraphics[width=.9\linewidth]{./pictures/out_b_100.png}
	\caption[Detection of buildings, example]{An example of the~detection; epoch
	100, loss function 0.8245}
      \label{fig:build-5}
\end{figure}

\begin{figure}[H]
   \centering
	\includegraphics[width=.9\linewidth]{./pictures/out_b_150.png}
	\caption[Detection of buildings, example]{An example of the~detection; epoch
	150, loss function 0.6280}
      \label{fig:build-6}
\end{figure}

\begin{figure}[H]
   \centering
	\includegraphics[width=.9\linewidth]{./pictures/out_b_180.png}
	\caption[Detection of buildings, example]{An example of the~detection; epoch
	180, loss function 0.5019}
      \label{fig:build-1}
\end{figure}

\chapter{E-attachments}
\label{attach}

E-attachment of this thesis consists of following features:

\begin{itemize}
	\item The source code of GRASS GIS modules and library
	\item GRASS GIS patch for Python 3
\end{itemize}


% konec dokumentu
\end{document}
