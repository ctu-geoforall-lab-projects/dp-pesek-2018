\chapter{Used technologies}
\label{technologies}

This chapter briefly introduces the most important technologies used during
the~development of modules for GRASS GIS. It means that besides GRASS GIS itself, 
the~Python language and its libraries TensorFlow and Keras will be introduced.

\section{GRASS GIS}
\label{grass}

\begin{figure}[H]
   \centering
	\includegraphics[width=0.4\linewidth]{./pictures/grass-logo.png}
	\caption[GRASS GIS logo]{GRASS GIS logo, source: \url{https://grass.osgeo.org/download/logos/}}
      \label{fig:grass-logo}
\end{figure}

The history of \zk{GRASS} \zk{GIS} (an acronym for Geographic Resources Analysis 
Support System Geographic Information System) started in year 1982 at the U.S. 
Army Corps of Engineers Construction Engineering Research Laboratory, but the 
version 1.0 under the name \zk{GRASS} was released later, in 1985. However, its 
way lead from the national project to the international Open GRASS Consortium in 
the mid 1990s. This consortium is perceived as an ancestor of today's Open 
Geospatial Consortium (\zk{OGC}). In 1999, \zk{GRASS} \zk{GIS} was published 
under GNU General Public License (\zk{GPL}).

\zk{GRASS} \zk{GIS} grown to a cross-platform free and open-source \zk{GIS} 
maintained by an international team of developers and users and licensed under 
the GNU \zk{GPL} license allowing users to perform geospatial data management 
and analysis for both raster and vector data, image processing, geocoding and 
visualization. Within its more than 300 modules, it supports also 
spatio-temporal data.

For more info about \zk{GRASS} \zk{GIS}, please see its original 
website\footnote{\url{https://grass.osgeo.org/documentation/general-overview/}}.

\section{Python}
\label{python}

\begin{figure}[H]
   \centering
	\includegraphics[width=\linewidth]{./pictures/python-logo.png}
	\caption[Python logo]{Python logo, source: \url{https://www.python.org/community/logos/}}
      \label{fig:python-logo}
\end{figure}

Python is a cross-platform open-source interpreted programming language. Python 
was invented by a~Dutch programmer and Monty Python's Flying Circus fan Guido 
van~Rossum in year 1991. Due to its user-friendly and readable structure, it is 
more and more popular.

It is very popular also in the field of open-source \zk{GIS}. Many modules for 
\zk{GRASS} \zk{GIS} are written in Python and Python is the most used language 
in QGIS\footnote{\url{https://qgis.org/}}.

For info about programming in Python, please see \cite{diveintopython}.

\section{TensorFlow}
\label{tf}

\begin{figure}[H]
   \centering
	\includegraphics[scale=0.5]{./pictures/tf-logo.png}
	\caption[TensorFlow logo]{TensorFlow logo, source: \url{https://www.tensorflow.org/images/tf_logo_transp.png}}
      \label{fig:tf-logo}
\end{figure}

TensorFlow is an open-source software library developed by Google Brain 
Team\footnote{\url{https://research.google.com/teams/brain/}} primarily for the 
purpose of neural network research. TesorFlow uses data flow graphs, where a 
node represent a mathematical operation and an edge represents a~tensor (a 
multidimensional array). A toolkit for visualization of these graphs is called 
TensorBoard.

TensorFlow is available for Python and C++ programming languages. Its usage 
optimizes mathematical expressions and as it was developed for the purpose of 
neural networks, the main focus was dedicated to this field and TensorFlow 
provides a lot of functions useful for deep learning.

A good starting point is book Getting Started with TensorFlow \cite{tf}. A 
documentation and examples can be found also at the original 
website\footnote{\url{https://www.tensorflow.org/}}.

\section{Keras}
\label{keras}

\begin{figure}[H]
   \centering
	\includegraphics[width=\linewidth]{./pictures/keras-logo.png}
	\caption[Keras logo]{Keras logo, source: \url{https://keras.io/}}
      \label{fig:keras-logo}
\end{figure}

Keras is an open-source software library written in Python and using TensorFlow, 
Microsoft Cognitive Toolkit, Theano or MXNet as a backend tensor manipulation 
library.

Keras is again developed for the purpose of deep learning. Its powerful feature 
is the content of extensible objects for defining different types of layers (see 
chapter \ref{layers}), models, loss functions and other tools widely used in the 
field of \zk{ANN}s.

For a wider documentation, please see the original 
website\footnote{\url{https://keras.io/}}.
