\chapter{Conclusion}
\label{conclusion}

The goal of the~thesis was to find and implement a~suitable \zk{ANN} 
architecture into \zk{GRASS} \zk{GIS}. Because \zk{ANN}s and especially 
\zk{CNN}s are shaking with the~field of computer vision, many GRASS GIS users 
and developers were interested in implementation options into this system.

The first part of the~thesis was dedicated to a~theoretical background behind 
\zk{CNN}s. It is followed by their various applications in the~field of computer 
vision. 

The second part of the~thesis was dedicated to the implementation of Mask
\zk{R-CNN} modules into GRASS GIS. It starts with a~brief introduction of some
of the~used tools and follows with explanations of the~most important parts of
the~code.

The research flows in the~background of the~theoretical part and is briefly 
summarized at the~beginning of the~chapter on the~implementation.

Developed modules are ready to use and as can be seen in appendices, I already 
made few tests. However, because the~training costs a~lot of time (at least few 
days on a~GPU machine, even weeks on a~machine without GPU), only a~limited 
amount of tests was made. Modules were proposed for testing also to other GRASS 
GIS developers and users and they found their results to be satisfying, 
sometimes even for data, where other methods of classification in GRASS GIS 
failed (a shadow over a~field, different colours, etc.).

Developed modules are available from the GitHub
repository\footnote{\url{https://github.com/ctu-geoforall-lab-projects/dp-pesek-2018}}.

Even though modules work, there are still some possible extensions and even
some issues.

The biggest issue is in insufficient support for Python 3 in GRASS GIS. This 
issue is solved by a~patch attached as an e-attachment and should be solved in 
GRASS GIS during this year.

Possible extensions are:
\begin{itemize}
	\item Support more training data annotation structures. The~current state
	works with \verb|*.jpg| images and \verb|*.png| masks for each instance.
	The~next step could be a~JSON-based structure used in MS
	COCO\footnote{\url{http://cocodataset.org/\#format-data}}.
	\item Support training on images in GRASS GIS (using raster map as images,
	vector areas as masks).
	\item Support more channels than three.
	\item Support more types of the~backbone architecture. Now only \zk{ResNet} 50
	and \zk{ResNet} 101 are supported.
	\item Diversify head architecture types. The~current one is just very basic
	one.
	\item Implement image augmentation for the training.
\end{itemize}

The last thing worthy of mention is the~high gear of progress. After only six 
years of research in the~field of \zk{CNN}s, black is white and up is down. 
Everything has changed and results unbelievable six years back are standards 
nowadays. Almost every week, there is a~new architecture. Even during work on 
this project, many interesting architectures were proposed, to name a~few, 
\cite{masklab} or \cite{panoptic}.

Developed modules are the~first step of \zk{GRASS} \zk{GIS} in the~direction to 
neural networks. Many functions could be useful for developing more modules. It 
would be pleasant to start a~new trend and see more modules using \zk{CNN}s and 
sharing libraries. Unsupervised learning is an open call.
